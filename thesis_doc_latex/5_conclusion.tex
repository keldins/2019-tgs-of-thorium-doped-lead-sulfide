%% This is an example first chapter.  You should put chapter/appendix that you
%% write into a separate file, and add a line \include{yourfilename} to
%% main.tex, where `yourfilename.tex' is the name of the chapter/appendix file.
%% You can process specific files by typing their names in at the 
%% \files=
%% prompt when you run the file main.tex through LaTeX.
\chapter{Conclusion}
The findings of this work, like so many, create more questions than they put to rest. 
SAW speed increases with radiation damage, and annealing can indeed heal this damage to restore SAW speed to that of the defect-free crystal. This makes for an unintentional confirmation of the findings of Dienes and Nabarro. It appears that thermal diffusivity, at least in lead sulfide thin films, also increases with radiation damage. This effect in particular warrants additional study, if only to determine the mechanism behind it.

Are new energy levels in the band gap liberating electrons in the short term, only to be swamped by the larger but latent effect of phonon scattering in the long term? Additional studies of radiation damage in PbS are necessary, wherein the sample is tested at logarithmically spaced time intervals. This could definitively rule out latent effects. A study of radiation damage's effect on band gap could also elucidate which new energy levels are available and how much they actually affect charge carrier mobility. An x-ray diffraction study could reveal the time evolution of the crystal lattice, which would measure more directly the amount of lattice amorphization that corresponds to the steady state damage condition this work was predicated on. 

Other potential follow up studies include both the simpler and the more complex. Affects of annealing and rapid pressure changes are unlikely but possible sources of effects on thermal diffusivity. It would be especially telling if a repetition of this experiment were carried out with several different annealing temperatures. One could rule out an affect of annealing temperature on elastic and thermal properties. An XRD study could potentially reveal if PbS, especially when doped with thorium, is cable of a previously unknown solid phase. Finally, a mathematically sophisticated but very necessary study must be carried out to deconvolve the respective contributions of the substrate and thin film to the signal captured by TGS when the grating spacing $\lambda$---and thus the interrogation depth---exceeds film thickness. Because quantum confinement is one most exciting features of thin films, future films to be studied with TGS will likely be thinner than 500 nm, not thicker.








%SAW speed increases with radiation damage
%Thermal diffusivity increases with radiation damage
%Does it come back down later?
%Did annealing cause a problem?
%Someone needs to follow up and do more TGS on aged sample
%Someone needs to do band gap measurements to realize true potential of self-irradiation
%Band gap changes might indicate any of these effects, especially the presence of additional energy levels in the gap.
%Someone needs to measure resistance



%Mike Short recommends these:
%G. J. Dienes, A theoretical estimate of the effect of radiation on the elastic constants of simple metals, Phys. Rev. 86, 228 (1952).
%F. R. N. Nabarro, Effect of radiation on elastic constants, Phys. Rev. 87, 665 (1952).
%G. J. Dienes, Effect of radiation on elastic constants, Phys. Rev. 87, 666 (1952).




%Although the findings laid out above are inconclusive in that (why are the inconclusive), they lay important groundwork for future iterations of these (describe tests). The results of my experiment indicate that (main finding), however in future experiments, researchers should be careful to (major thing that should be done differently next time). 
%(Thing I'm suggesting researches do differently next time) could lead to (suggest a finding if researchers follow your advice).
%(Another suggestion of what you think could be done differently next time) (A final suggestion of what you think could be done differently next time) 
%(A sentence stating your main finding and encouraging scientists to pick up where you left off.) (A sentence reiterating the limitations of your finding)


% \cite{patterson:risc,rad83}.  
% \cite{ellis:bulldog,pet87,coutant:precision-compilers}.  In these cases, the
% \cite{gib86}.
% These optimizations are described in detail in section~\ref{ch1:opts}.
% \section{Description of micro-optimization}\label{ch1:opts}
% \footnote{A description of the floating point format used is shown in figures~\ref{exponent-format}
% and~\ref{mantissa-format}.}.  
% A discussion of the mathematics behind unnormalized arithmetic is in appendix~\ref{unnorm-math}.

% \footnote{Using unnormalized numbers for math is not a new idea; a
% good example of it is the Control Data CDC 6600, designed by Seymour Cray.
% \cite{thornton:cdc6600} The CDC 6600 had all of its instructions performing
% unnormalized arithmetic, with a separate {\tt NORMALIZE} instruction.}.

% This is an example of how you would use tgrind to include an example
% of source code; it is commented out in this template since the code
% example file does not exist.  To use it, you need to remove the '%' on the
% beginning of the line, and insert your own information in the call.
%
%\tagrind[htbp]{code/pmn.s.tex}{Post Multiply Normalization}{opt:pmn}

% This is an example of how you would use tgrind to include an example
% of source code; it is commented out in this template since the code
% example file does not exist.  To use it, you need to remove the '%' on the
% beginning of the line, and insert your own information in the call.
%
%\tgrind[htbp]{code/be.s.tex}{Block Exponent}{opt:be}
