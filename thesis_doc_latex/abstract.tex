% $Log: abstract.tex,v $
% Revision 1.1  93/05/14  14:56:25  starflt
% Initial revision
% 
% Revision 1.1  90/05/04  10:41:01  lwvanels
% Initial revision
% 
%
%% The text of your abstract and nothing else (other than comments) goes here.
%% It will be single-spaced and the rest of the text that is supposed to go on
%% the abstract page will be generated by the abstractpage environment.  This
%% file should be \input (not \include 'd) from cover.tex.
Lead sulfide (PbS) is an important semiconductor for infrared light detection, and use in space necessitates understanding how it evolves when damaged by ionizing radiation. Previous work in chemical bath deposition (CBD) resulted in thin films of epitaxially grown polycrystalline PbS uniformly doped with radioactive thorium 228 (Th-228), permitting convenient study of a self-irradiating sample. This thesis represents a continuation of that work by studying the evolution of thermal diffusivity and surface acoustic wave (SAW) speed in a self-irradiating PbS thin film using the non-contact, non-destructive transient grating spectroscopy (TGS) assay. Radiation damage is allowed to accumulate and TGS is used to take measurements before and after annealing. Damage was presumed to create new phonon-scattering defects, thus decreasing SAW speed and thermal diffusivity. However, after annealing, radiation damage caused a monotonic increase in both. Both parameters asymptotically approach a maximum, which indicates a radiation damage saturation point. Thermal diffusivity does not return to its pre-annealed value, indicating an unknown affect. A longer TGS study is recommended to eliminate latent effects, as well as a band gap time-evolution study and an x-ray diffraction study.
